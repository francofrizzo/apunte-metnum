% -*- root: apunte-metodos.tex -*-

\section{Integración numérica}

El cálculo de \textbf{integrales} es un problema matemático muy común, que
tiene multitud de aplicaciones en diversos campos de la ciencia y de la
ingeniería: cálculo de áreas y volúmenes, obtención de ciertas magnitudes
físicas a partir de otras (desplazamiento a partir de la velocidad, por
ejemplo), etc. La forma clásica de calcular integrales es analítica: se busca
una primitiva de la función a integrar, que luego se evalúa en sus dos
extremos. Sin embargo, esta solución no siempre es factible. Por ejemplo,
existen funciones cuya integral analítica es una expresión complicada de
obtener o de evaluar computacionalmente, o incluso algunas para las que no
existe una expresión analítica conocida. Peor aún, a veces es necesario
integrar funciones de las que se conocen únicamente algunos valores (por
ejemplo, muestras obtenidos empíricamente), volviendo imposible la integración
analítica.

En estos casos, es útil contar con métodos que permitan aproximar el valor de
una integral de manera numérical; esto en general se hace computando una suma
finita de valores, que pueden ser integrales de funciones más sencillas, por
medio de lo que se conoce como \textbf{fórmula de cuadratura}. Una opción muy
utilizada es construir fórmulas de cuadratura a partir de polinomios
interpoladores, ya que constituyen una buena aproximación de la función a
integrar, sus integrales son sencillas de calcular analíticamente y pueden
ser evaluadas en forma eficiente, y permiten calcular una cota para el
error cometido en la aproximación.

\subsection{Regla del trapecio}

Consideremos una función $f \in \mathcal{C}^2([a,b])$, para la cual queremos
calcular
\[ \int_a^b f(x) \,dx \]
para la que queremos derivar una fórmula de cuadratura a partir del polinomio
interpolador de Lagrange en los puntos $x_0 = a$ y $x_1 = b$. Escribimos $f(x)
= P(x) + E(x)$, donde $P(x)$ es este polinomio y $E(x)$ representa el error de
aproximación para cada punto. Así,
\[ P(x) =  f(x_0) \cdot \frac{x - x_1}{x_0 - x_1}
    + f(x_1) \cdot \frac{x - x_0}{x_1 - x_0}
    \qquad \text{y} \qquad
    E(x) = \frac{f''(\xi_x)}{2} \cdot (x-x_0) \cdot (x-x_1) \]

Por un lado,
\[ \begin{aligned} \int_{x_0}^{x_1} P(x) \,dx
    &= \int_{x_0}^{x_1} f(x_0) \cdot \frac{x - x_1}{x_0 - x_1}
        + f(x_1) \cdot \frac{x - x_0}{x_1 - x_0} \,dx \\
    &= \left. \left[ f(x_0) \cdot \frac{(x - x_1)^2}{2 \cdot (x_0 - x_1)} + f(x_1) \cdot
        \frac{(x - x_0)^2}{2 \cdot (x_1 - x_0)} \right] \right\rvert_{x_0}^{x_1} \\
    &= \frac{x_1 - x_0}{2} \cdot \left[ f(x_1) + f(x_0) \right]
    \end{aligned} \]

Por otra parte,
\[ \begin{aligned} \int_{x_0}^{x_1} E(x) \,dx
    &= \int_{x_0}^{x_1} \frac{f''(\xi_x)}{2} \cdot (x-x_0) \cdot (x-x_1) \,dx
        && \text{para algún $\xi_x \in (x_0,x_1)$} \\
    &= \frac{f''(\xi)}{2} \cdot \int_{x_0}^{x_1} (x-x_0) \cdot (x-x_1) \,dx
        && \text{para algún $\xi \in (x_0,x_1)$} \\
    &= \frac{f''(\xi)}{2} \cdot \left. \left[ \frac{x^3}{3} - \frac{x_1 + x_0}{2}
        \cdot x^2 + x_0 \cdot x_1 \cdot x \right] \right\rvert_{x_0}^{x_1} \\
    &= - \frac{(x_1 - x_0)^3}{12} \cdot f''(\xi)
    \end{aligned} \]

Entonces, llamando $h = x_1 - x_0$, tenemos

\[ \begin{aligned} \int_a^b f(x) \,dx
    &= \int_{x_0}^{x_1} P(x) \,dx + \int_{x_0}^{x_1} E(x) \,dx \\
    &= \frac{h}{2} \cdot \left[ f(x_1) + f(x_0) \right] - \frac{h^3}{12} \cdot f''(\xi)
    \end{aligned} \]

Esta formulación de la integral de $f$ recibe la denominación de \textbf{regla del
trapecio} debido a que, cuando los valores de $f(x_0)$ y $f(x_1)$ tienen igual
signo, se la puede interpretar gráficamente como el cálculo del área del
trapecio que queda bajo la recta que los une, como puede verse en la
Figura (fig).

(GRÁFICO DE LA REGLA DEL TRAPECIO)

Podemos notar que el término del error está multiplicado por la segunda
derivada de $f$. Por este motivo, si $f$ es tal que $f''(x)$ se anula en todo
punto, es decir, si se trata de un polinomio de grado menor o igual que $1$,
el término del error es idénticamente $0$ y la aproximación obtenida para la
integral es exacta.

\subsection{Regla de Simpson}


\subsection{Métodos adaptativos}

% -*- root: apunte-metodos.tex -*-

\section{Cálculo de autovalores}
\subsection{Método de la potencia}

Del TP:
\emph{método de la potencia}. Este método se basa en el siguiente resultado:
si $x$ es el autovector principal de una matriz $\mat{A}$, y $v$ es un vector
inicial cualquiera, entonces $\lim_{k\to\infty}\mat{A}^k v = x$. Es decir,
partiendo de un vector inicial y multiplicando repetidas veces por la matriz
$\mat{A}$, el resultado eventualmente convergerá al autovector buscado. Para
una demostración de este hecho, como así también una exposición más detallada
del proceso, puede consultarse: Kamvar, Sepandar D. and Haveliwala, Taher H.
and Manning, Christopher D. and Golub, Gene H.: \emph{Extrapolation
methods for accelerating PageRank computations}.

\subsection{Método de la potencia inversa}
